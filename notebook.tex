
% Default to the notebook output style

    


% Inherit from the specified cell style.




    
\documentclass[11pt]{article}

    
    
    \usepackage[T1]{fontenc}
    % Nicer default font (+ math font) than Computer Modern for most use cases
    \usepackage{mathpazo}

    % Basic figure setup, for now with no caption control since it's done
    % automatically by Pandoc (which extracts ![](path) syntax from Markdown).
    \usepackage{graphicx}
    % We will generate all images so they have a width \maxwidth. This means
    % that they will get their normal width if they fit onto the page, but
    % are scaled down if they would overflow the margins.
    \makeatletter
    \def\maxwidth{\ifdim\Gin@nat@width>\linewidth\linewidth
    \else\Gin@nat@width\fi}
    \makeatother
    \let\Oldincludegraphics\includegraphics
    % Set max figure width to be 80% of text width, for now hardcoded.
    \renewcommand{\includegraphics}[1]{\Oldincludegraphics[width=.8\maxwidth]{#1}}
    % Ensure that by default, figures have no caption (until we provide a
    % proper Figure object with a Caption API and a way to capture that
    % in the conversion process - todo).
    \usepackage{caption}
    \DeclareCaptionLabelFormat{nolabel}{}
    \captionsetup{labelformat=nolabel}

    \usepackage{adjustbox} % Used to constrain images to a maximum size 
    \usepackage{xcolor} % Allow colors to be defined
    \usepackage{enumerate} % Needed for markdown enumerations to work
    \usepackage{geometry} % Used to adjust the document margins
    \usepackage{amsmath} % Equations
    \usepackage{amssymb} % Equations
    \usepackage{textcomp} % defines textquotesingle
    % Hack from http://tex.stackexchange.com/a/47451/13684:
    \AtBeginDocument{%
        \def\PYZsq{\textquotesingle}% Upright quotes in Pygmentized code
    }
    \usepackage{upquote} % Upright quotes for verbatim code
    \usepackage{eurosym} % defines \euro
    \usepackage[mathletters]{ucs} % Extended unicode (utf-8) support
    \usepackage[utf8x]{inputenc} % Allow utf-8 characters in the tex document
    \usepackage{fancyvrb} % verbatim replacement that allows latex
    \usepackage{grffile} % extends the file name processing of package graphics 
                         % to support a larger range 
    % The hyperref package gives us a pdf with properly built
    % internal navigation ('pdf bookmarks' for the table of contents,
    % internal cross-reference links, web links for URLs, etc.)
    \usepackage{hyperref}
    \usepackage{longtable} % longtable support required by pandoc >1.10
    \usepackage{booktabs}  % table support for pandoc > 1.12.2
    \usepackage[inline]{enumitem} % IRkernel/repr support (it uses the enumerate* environment)
    \usepackage[normalem]{ulem} % ulem is needed to support strikethroughs (\sout)
                                % normalem makes italics be italics, not underlines
    

    
    
    % Colors for the hyperref package
    \definecolor{urlcolor}{rgb}{0,.145,.698}
    \definecolor{linkcolor}{rgb}{.71,0.21,0.01}
    \definecolor{citecolor}{rgb}{.12,.54,.11}

    % ANSI colors
    \definecolor{ansi-black}{HTML}{3E424D}
    \definecolor{ansi-black-intense}{HTML}{282C36}
    \definecolor{ansi-red}{HTML}{E75C58}
    \definecolor{ansi-red-intense}{HTML}{B22B31}
    \definecolor{ansi-green}{HTML}{00A250}
    \definecolor{ansi-green-intense}{HTML}{007427}
    \definecolor{ansi-yellow}{HTML}{DDB62B}
    \definecolor{ansi-yellow-intense}{HTML}{B27D12}
    \definecolor{ansi-blue}{HTML}{208FFB}
    \definecolor{ansi-blue-intense}{HTML}{0065CA}
    \definecolor{ansi-magenta}{HTML}{D160C4}
    \definecolor{ansi-magenta-intense}{HTML}{A03196}
    \definecolor{ansi-cyan}{HTML}{60C6C8}
    \definecolor{ansi-cyan-intense}{HTML}{258F8F}
    \definecolor{ansi-white}{HTML}{C5C1B4}
    \definecolor{ansi-white-intense}{HTML}{A1A6B2}

    % commands and environments needed by pandoc snippets
    % extracted from the output of `pandoc -s`
    \providecommand{\tightlist}{%
      \setlength{\itemsep}{0pt}\setlength{\parskip}{0pt}}
    \DefineVerbatimEnvironment{Highlighting}{Verbatim}{commandchars=\\\{\}}
    % Add ',fontsize=\small' for more characters per line
    \newenvironment{Shaded}{}{}
    \newcommand{\KeywordTok}[1]{\textcolor[rgb]{0.00,0.44,0.13}{\textbf{{#1}}}}
    \newcommand{\DataTypeTok}[1]{\textcolor[rgb]{0.56,0.13,0.00}{{#1}}}
    \newcommand{\DecValTok}[1]{\textcolor[rgb]{0.25,0.63,0.44}{{#1}}}
    \newcommand{\BaseNTok}[1]{\textcolor[rgb]{0.25,0.63,0.44}{{#1}}}
    \newcommand{\FloatTok}[1]{\textcolor[rgb]{0.25,0.63,0.44}{{#1}}}
    \newcommand{\CharTok}[1]{\textcolor[rgb]{0.25,0.44,0.63}{{#1}}}
    \newcommand{\StringTok}[1]{\textcolor[rgb]{0.25,0.44,0.63}{{#1}}}
    \newcommand{\CommentTok}[1]{\textcolor[rgb]{0.38,0.63,0.69}{\textit{{#1}}}}
    \newcommand{\OtherTok}[1]{\textcolor[rgb]{0.00,0.44,0.13}{{#1}}}
    \newcommand{\AlertTok}[1]{\textcolor[rgb]{1.00,0.00,0.00}{\textbf{{#1}}}}
    \newcommand{\FunctionTok}[1]{\textcolor[rgb]{0.02,0.16,0.49}{{#1}}}
    \newcommand{\RegionMarkerTok}[1]{{#1}}
    \newcommand{\ErrorTok}[1]{\textcolor[rgb]{1.00,0.00,0.00}{\textbf{{#1}}}}
    \newcommand{\NormalTok}[1]{{#1}}
    
    % Additional commands for more recent versions of Pandoc
    \newcommand{\ConstantTok}[1]{\textcolor[rgb]{0.53,0.00,0.00}{{#1}}}
    \newcommand{\SpecialCharTok}[1]{\textcolor[rgb]{0.25,0.44,0.63}{{#1}}}
    \newcommand{\VerbatimStringTok}[1]{\textcolor[rgb]{0.25,0.44,0.63}{{#1}}}
    \newcommand{\SpecialStringTok}[1]{\textcolor[rgb]{0.73,0.40,0.53}{{#1}}}
    \newcommand{\ImportTok}[1]{{#1}}
    \newcommand{\DocumentationTok}[1]{\textcolor[rgb]{0.73,0.13,0.13}{\textit{{#1}}}}
    \newcommand{\AnnotationTok}[1]{\textcolor[rgb]{0.38,0.63,0.69}{\textbf{\textit{{#1}}}}}
    \newcommand{\CommentVarTok}[1]{\textcolor[rgb]{0.38,0.63,0.69}{\textbf{\textit{{#1}}}}}
    \newcommand{\VariableTok}[1]{\textcolor[rgb]{0.10,0.09,0.49}{{#1}}}
    \newcommand{\ControlFlowTok}[1]{\textcolor[rgb]{0.00,0.44,0.13}{\textbf{{#1}}}}
    \newcommand{\OperatorTok}[1]{\textcolor[rgb]{0.40,0.40,0.40}{{#1}}}
    \newcommand{\BuiltInTok}[1]{{#1}}
    \newcommand{\ExtensionTok}[1]{{#1}}
    \newcommand{\PreprocessorTok}[1]{\textcolor[rgb]{0.74,0.48,0.00}{{#1}}}
    \newcommand{\AttributeTok}[1]{\textcolor[rgb]{0.49,0.56,0.16}{{#1}}}
    \newcommand{\InformationTok}[1]{\textcolor[rgb]{0.38,0.63,0.69}{\textbf{\textit{{#1}}}}}
    \newcommand{\WarningTok}[1]{\textcolor[rgb]{0.38,0.63,0.69}{\textbf{\textit{{#1}}}}}
    
    
    % Define a nice break command that doesn't care if a line doesn't already
    % exist.
    \def\br{\hspace*{\fill} \\* }
    % Math Jax compatability definitions
    \def\gt{>}
    \def\lt{<}
    % Document parameters
    \title{Tarea\_Colas}
    
    
    

    % Pygments definitions
    
\makeatletter
\def\PY@reset{\let\PY@it=\relax \let\PY@bf=\relax%
    \let\PY@ul=\relax \let\PY@tc=\relax%
    \let\PY@bc=\relax \let\PY@ff=\relax}
\def\PY@tok#1{\csname PY@tok@#1\endcsname}
\def\PY@toks#1+{\ifx\relax#1\empty\else%
    \PY@tok{#1}\expandafter\PY@toks\fi}
\def\PY@do#1{\PY@bc{\PY@tc{\PY@ul{%
    \PY@it{\PY@bf{\PY@ff{#1}}}}}}}
\def\PY#1#2{\PY@reset\PY@toks#1+\relax+\PY@do{#2}}

\expandafter\def\csname PY@tok@gd\endcsname{\def\PY@tc##1{\textcolor[rgb]{0.63,0.00,0.00}{##1}}}
\expandafter\def\csname PY@tok@gu\endcsname{\let\PY@bf=\textbf\def\PY@tc##1{\textcolor[rgb]{0.50,0.00,0.50}{##1}}}
\expandafter\def\csname PY@tok@gt\endcsname{\def\PY@tc##1{\textcolor[rgb]{0.00,0.27,0.87}{##1}}}
\expandafter\def\csname PY@tok@gs\endcsname{\let\PY@bf=\textbf}
\expandafter\def\csname PY@tok@gr\endcsname{\def\PY@tc##1{\textcolor[rgb]{1.00,0.00,0.00}{##1}}}
\expandafter\def\csname PY@tok@cm\endcsname{\let\PY@it=\textit\def\PY@tc##1{\textcolor[rgb]{0.25,0.50,0.50}{##1}}}
\expandafter\def\csname PY@tok@vg\endcsname{\def\PY@tc##1{\textcolor[rgb]{0.10,0.09,0.49}{##1}}}
\expandafter\def\csname PY@tok@vi\endcsname{\def\PY@tc##1{\textcolor[rgb]{0.10,0.09,0.49}{##1}}}
\expandafter\def\csname PY@tok@vm\endcsname{\def\PY@tc##1{\textcolor[rgb]{0.10,0.09,0.49}{##1}}}
\expandafter\def\csname PY@tok@mh\endcsname{\def\PY@tc##1{\textcolor[rgb]{0.40,0.40,0.40}{##1}}}
\expandafter\def\csname PY@tok@cs\endcsname{\let\PY@it=\textit\def\PY@tc##1{\textcolor[rgb]{0.25,0.50,0.50}{##1}}}
\expandafter\def\csname PY@tok@ge\endcsname{\let\PY@it=\textit}
\expandafter\def\csname PY@tok@vc\endcsname{\def\PY@tc##1{\textcolor[rgb]{0.10,0.09,0.49}{##1}}}
\expandafter\def\csname PY@tok@il\endcsname{\def\PY@tc##1{\textcolor[rgb]{0.40,0.40,0.40}{##1}}}
\expandafter\def\csname PY@tok@go\endcsname{\def\PY@tc##1{\textcolor[rgb]{0.53,0.53,0.53}{##1}}}
\expandafter\def\csname PY@tok@cp\endcsname{\def\PY@tc##1{\textcolor[rgb]{0.74,0.48,0.00}{##1}}}
\expandafter\def\csname PY@tok@gi\endcsname{\def\PY@tc##1{\textcolor[rgb]{0.00,0.63,0.00}{##1}}}
\expandafter\def\csname PY@tok@gh\endcsname{\let\PY@bf=\textbf\def\PY@tc##1{\textcolor[rgb]{0.00,0.00,0.50}{##1}}}
\expandafter\def\csname PY@tok@ni\endcsname{\let\PY@bf=\textbf\def\PY@tc##1{\textcolor[rgb]{0.60,0.60,0.60}{##1}}}
\expandafter\def\csname PY@tok@nl\endcsname{\def\PY@tc##1{\textcolor[rgb]{0.63,0.63,0.00}{##1}}}
\expandafter\def\csname PY@tok@nn\endcsname{\let\PY@bf=\textbf\def\PY@tc##1{\textcolor[rgb]{0.00,0.00,1.00}{##1}}}
\expandafter\def\csname PY@tok@no\endcsname{\def\PY@tc##1{\textcolor[rgb]{0.53,0.00,0.00}{##1}}}
\expandafter\def\csname PY@tok@na\endcsname{\def\PY@tc##1{\textcolor[rgb]{0.49,0.56,0.16}{##1}}}
\expandafter\def\csname PY@tok@nb\endcsname{\def\PY@tc##1{\textcolor[rgb]{0.00,0.50,0.00}{##1}}}
\expandafter\def\csname PY@tok@nc\endcsname{\let\PY@bf=\textbf\def\PY@tc##1{\textcolor[rgb]{0.00,0.00,1.00}{##1}}}
\expandafter\def\csname PY@tok@nd\endcsname{\def\PY@tc##1{\textcolor[rgb]{0.67,0.13,1.00}{##1}}}
\expandafter\def\csname PY@tok@ne\endcsname{\let\PY@bf=\textbf\def\PY@tc##1{\textcolor[rgb]{0.82,0.25,0.23}{##1}}}
\expandafter\def\csname PY@tok@nf\endcsname{\def\PY@tc##1{\textcolor[rgb]{0.00,0.00,1.00}{##1}}}
\expandafter\def\csname PY@tok@si\endcsname{\let\PY@bf=\textbf\def\PY@tc##1{\textcolor[rgb]{0.73,0.40,0.53}{##1}}}
\expandafter\def\csname PY@tok@s2\endcsname{\def\PY@tc##1{\textcolor[rgb]{0.73,0.13,0.13}{##1}}}
\expandafter\def\csname PY@tok@nt\endcsname{\let\PY@bf=\textbf\def\PY@tc##1{\textcolor[rgb]{0.00,0.50,0.00}{##1}}}
\expandafter\def\csname PY@tok@nv\endcsname{\def\PY@tc##1{\textcolor[rgb]{0.10,0.09,0.49}{##1}}}
\expandafter\def\csname PY@tok@s1\endcsname{\def\PY@tc##1{\textcolor[rgb]{0.73,0.13,0.13}{##1}}}
\expandafter\def\csname PY@tok@dl\endcsname{\def\PY@tc##1{\textcolor[rgb]{0.73,0.13,0.13}{##1}}}
\expandafter\def\csname PY@tok@ch\endcsname{\let\PY@it=\textit\def\PY@tc##1{\textcolor[rgb]{0.25,0.50,0.50}{##1}}}
\expandafter\def\csname PY@tok@m\endcsname{\def\PY@tc##1{\textcolor[rgb]{0.40,0.40,0.40}{##1}}}
\expandafter\def\csname PY@tok@gp\endcsname{\let\PY@bf=\textbf\def\PY@tc##1{\textcolor[rgb]{0.00,0.00,0.50}{##1}}}
\expandafter\def\csname PY@tok@sh\endcsname{\def\PY@tc##1{\textcolor[rgb]{0.73,0.13,0.13}{##1}}}
\expandafter\def\csname PY@tok@ow\endcsname{\let\PY@bf=\textbf\def\PY@tc##1{\textcolor[rgb]{0.67,0.13,1.00}{##1}}}
\expandafter\def\csname PY@tok@sx\endcsname{\def\PY@tc##1{\textcolor[rgb]{0.00,0.50,0.00}{##1}}}
\expandafter\def\csname PY@tok@bp\endcsname{\def\PY@tc##1{\textcolor[rgb]{0.00,0.50,0.00}{##1}}}
\expandafter\def\csname PY@tok@c1\endcsname{\let\PY@it=\textit\def\PY@tc##1{\textcolor[rgb]{0.25,0.50,0.50}{##1}}}
\expandafter\def\csname PY@tok@fm\endcsname{\def\PY@tc##1{\textcolor[rgb]{0.00,0.00,1.00}{##1}}}
\expandafter\def\csname PY@tok@o\endcsname{\def\PY@tc##1{\textcolor[rgb]{0.40,0.40,0.40}{##1}}}
\expandafter\def\csname PY@tok@kc\endcsname{\let\PY@bf=\textbf\def\PY@tc##1{\textcolor[rgb]{0.00,0.50,0.00}{##1}}}
\expandafter\def\csname PY@tok@c\endcsname{\let\PY@it=\textit\def\PY@tc##1{\textcolor[rgb]{0.25,0.50,0.50}{##1}}}
\expandafter\def\csname PY@tok@mf\endcsname{\def\PY@tc##1{\textcolor[rgb]{0.40,0.40,0.40}{##1}}}
\expandafter\def\csname PY@tok@err\endcsname{\def\PY@bc##1{\setlength{\fboxsep}{0pt}\fcolorbox[rgb]{1.00,0.00,0.00}{1,1,1}{\strut ##1}}}
\expandafter\def\csname PY@tok@mb\endcsname{\def\PY@tc##1{\textcolor[rgb]{0.40,0.40,0.40}{##1}}}
\expandafter\def\csname PY@tok@ss\endcsname{\def\PY@tc##1{\textcolor[rgb]{0.10,0.09,0.49}{##1}}}
\expandafter\def\csname PY@tok@sr\endcsname{\def\PY@tc##1{\textcolor[rgb]{0.73,0.40,0.53}{##1}}}
\expandafter\def\csname PY@tok@mo\endcsname{\def\PY@tc##1{\textcolor[rgb]{0.40,0.40,0.40}{##1}}}
\expandafter\def\csname PY@tok@kd\endcsname{\let\PY@bf=\textbf\def\PY@tc##1{\textcolor[rgb]{0.00,0.50,0.00}{##1}}}
\expandafter\def\csname PY@tok@mi\endcsname{\def\PY@tc##1{\textcolor[rgb]{0.40,0.40,0.40}{##1}}}
\expandafter\def\csname PY@tok@kn\endcsname{\let\PY@bf=\textbf\def\PY@tc##1{\textcolor[rgb]{0.00,0.50,0.00}{##1}}}
\expandafter\def\csname PY@tok@cpf\endcsname{\let\PY@it=\textit\def\PY@tc##1{\textcolor[rgb]{0.25,0.50,0.50}{##1}}}
\expandafter\def\csname PY@tok@kr\endcsname{\let\PY@bf=\textbf\def\PY@tc##1{\textcolor[rgb]{0.00,0.50,0.00}{##1}}}
\expandafter\def\csname PY@tok@s\endcsname{\def\PY@tc##1{\textcolor[rgb]{0.73,0.13,0.13}{##1}}}
\expandafter\def\csname PY@tok@kp\endcsname{\def\PY@tc##1{\textcolor[rgb]{0.00,0.50,0.00}{##1}}}
\expandafter\def\csname PY@tok@w\endcsname{\def\PY@tc##1{\textcolor[rgb]{0.73,0.73,0.73}{##1}}}
\expandafter\def\csname PY@tok@kt\endcsname{\def\PY@tc##1{\textcolor[rgb]{0.69,0.00,0.25}{##1}}}
\expandafter\def\csname PY@tok@sc\endcsname{\def\PY@tc##1{\textcolor[rgb]{0.73,0.13,0.13}{##1}}}
\expandafter\def\csname PY@tok@sb\endcsname{\def\PY@tc##1{\textcolor[rgb]{0.73,0.13,0.13}{##1}}}
\expandafter\def\csname PY@tok@sa\endcsname{\def\PY@tc##1{\textcolor[rgb]{0.73,0.13,0.13}{##1}}}
\expandafter\def\csname PY@tok@k\endcsname{\let\PY@bf=\textbf\def\PY@tc##1{\textcolor[rgb]{0.00,0.50,0.00}{##1}}}
\expandafter\def\csname PY@tok@se\endcsname{\let\PY@bf=\textbf\def\PY@tc##1{\textcolor[rgb]{0.73,0.40,0.13}{##1}}}
\expandafter\def\csname PY@tok@sd\endcsname{\let\PY@it=\textit\def\PY@tc##1{\textcolor[rgb]{0.73,0.13,0.13}{##1}}}

\def\PYZbs{\char`\\}
\def\PYZus{\char`\_}
\def\PYZob{\char`\{}
\def\PYZcb{\char`\}}
\def\PYZca{\char`\^}
\def\PYZam{\char`\&}
\def\PYZlt{\char`\<}
\def\PYZgt{\char`\>}
\def\PYZsh{\char`\#}
\def\PYZpc{\char`\%}
\def\PYZdl{\char`\$}
\def\PYZhy{\char`\-}
\def\PYZsq{\char`\'}
\def\PYZdq{\char`\"}
\def\PYZti{\char`\~}
% for compatibility with earlier versions
\def\PYZat{@}
\def\PYZlb{[}
\def\PYZrb{]}
\makeatother


    % Exact colors from NB
    \definecolor{incolor}{rgb}{0.0, 0.0, 0.5}
    \definecolor{outcolor}{rgb}{0.545, 0.0, 0.0}



    
    % Prevent overflowing lines due to hard-to-break entities
    \sloppy 
    % Setup hyperref package
    \hypersetup{
      breaklinks=true,  % so long urls are correctly broken across lines
      colorlinks=true,
      urlcolor=urlcolor,
      linkcolor=linkcolor,
      citecolor=citecolor,
      }
    % Slightly bigger margins than the latex defaults
    
    \geometry{verbose,tmargin=1in,bmargin=1in,lmargin=1in,rmargin=1in}
    
    

    \begin{document}
    
    
    \maketitle
    
    

    
    \section{Round Robin:}\label{round-robin}

Este rograma que simula la ejecución de n procesos en un procesador de
un solo núcleo mediante el algorithmo Round Robin, en el que es posible
establecer el quantum (en segundos) y el tiempo que demora cada proceso
es un aleatorio entre 1 y 100 (segundos). Este muestra el tiempo de cada
proceso, los turnos de atención en el procesador y el número de turnos
totales.

    \subsection{Propósito}\label{propuxf3sito}

Un sistema de monoprogramación puede atender a un único proceso en
determinado tiempo. Con el fin de emular multiprogramación con un sólo
procesador es necesario porcionar el tiempo de CPU en un tamaño
determinado llamado quantum y atender un proceso únicamente durante el
quantum, a cuya terminación el programa es interrumpido y se atiende el
siguiente proceso. Este procedimiento se repite hasta que se finalicen
todos los procesos.

Para atender a los distintos procesos es posible asignar prioridad a
estos a partir del tiempo de CPU que requieren para ser culminados,
brindando la mayor prioridad a los procesos más pequeños, para evitar la
innanición. Sin embargo, el algoritmo Round Robin hace posible esta
emulación de multiprogramación sin la necesidad de asignar prioridad a
ninguno de los procesos, sin riesgo de innanición.

Este programa se realizó con el fin de ilustrar el funcionamiento del
algoritmo Round Robin, mostrando la forma en que se maneja la atención
de todos los procesos mediante turnos y el tiempo total de
procesamiento.

    \subsection{Programa}\label{programa}

\paragraph{Diagrama Flujo Nivel}\label{diagrama-flujo-nivel}

    \begin{Verbatim}[commandchars=\\\{\}]
{\color{incolor}In [{\color{incolor}5}]:} \PY{n}{resultados} \PY{o}{=} \PY{p}{[}\PY{p}{]}
\end{Verbatim}


    \begin{Verbatim}[commandchars=\\\{\}]
{\color{incolor}In [{\color{incolor}6}]:} \PY{k+kn}{import} \PY{n+nn}{numpy} \PY{k+kn}{as} \PY{n+nn}{np}
        
        \PY{k}{class} \PY{n+nc}{Proceso}\PY{p}{:}
            \PY{c+c1}{\PYZsh{}IdProc = 0}
            \PY{k}{def} \PY{n+nf+fm}{\PYZus{}\PYZus{}init\PYZus{}\PYZus{}}\PY{p}{(}\PY{n+nb+bp}{self}\PY{p}{,} \PY{n}{i}\PY{p}{,} \PY{n}{tiempo}\PY{p}{,} \PY{n}{estado}\PY{p}{)}\PY{p}{:}
                \PY{c+c1}{\PYZsh{}Proceso.IdProc += 1}
                \PY{c+c1}{\PYZsh{}self.IdProc = Proceso.IdProc}
                \PY{n+nb+bp}{self}\PY{o}{.}\PY{n}{IdProc} \PY{o}{=} \PY{n}{i}
                \PY{n+nb+bp}{self}\PY{o}{.}\PY{n}{tiempo} \PY{o}{=} \PY{n}{tiempo}
        
            \PY{k}{def} \PY{n+nf}{getIdProc}\PY{p}{(}\PY{n+nb+bp}{self}\PY{p}{)}\PY{p}{:}
                \PY{k}{return} \PY{n+nb+bp}{self}\PY{o}{.}\PY{n}{IdProc}
        
            \PY{k}{def} \PY{n+nf}{setTiempo}\PY{p}{(}\PY{n+nb+bp}{self}\PY{p}{,} \PY{n}{tiempo}\PY{p}{)}\PY{p}{:}
                \PY{n+nb+bp}{self}\PY{o}{.}\PY{n}{tiempo} \PY{o}{=} \PY{n}{tiempo}
        
            \PY{k}{def} \PY{n+nf}{getTiempo}\PY{p}{(}\PY{n+nb+bp}{self}\PY{p}{)}\PY{p}{:}
                \PY{k}{return} \PY{n+nb+bp}{self}\PY{o}{.}\PY{n}{tiempo}
        
            \PY{k}{def} \PY{n+nf}{ejecutar}\PY{p}{(}\PY{n+nb+bp}{self}\PY{p}{,} \PY{n}{q}\PY{p}{)}\PY{p}{:}
                \PY{n}{temp} \PY{o}{=} \PY{n+nb+bp}{self}\PY{o}{.}\PY{n}{tiempo}
                \PY{n+nb+bp}{self}\PY{o}{.}\PY{n}{tiempo} \PY{o}{\PYZhy{}}\PY{o}{=} \PY{n}{q}
                \PY{k}{if} \PY{n+nb+bp}{self}\PY{o}{.}\PY{n}{tiempo} \PY{o}{\PYZlt{}} \PY{l+m+mi}{0}\PY{p}{:}
                    \PY{n+nb+bp}{self}\PY{o}{.}\PY{n}{tiempo} \PY{o}{=} \PY{l+m+mi}{0}
                    \PY{k}{return} \PY{n}{temp}
                \PY{k}{else}\PY{p}{:}
                    \PY{k}{return} \PY{n}{q}
        
        \PY{k}{class} \PY{n+nc}{RoundRobin}\PY{p}{:}
            \PY{k}{def} \PY{n+nf+fm}{\PYZus{}\PYZus{}init\PYZus{}\PYZus{}}\PY{p}{(}\PY{n+nb+bp}{self}\PY{p}{,} \PY{n}{quantum}\PY{p}{,} \PY{n}{numProc}\PY{p}{)}\PY{p}{:}
                \PY{n+nb+bp}{self}\PY{o}{.}\PY{n}{quantum} \PY{o}{=} \PY{n}{quantum}
                \PY{n+nb+bp}{self}\PY{o}{.}\PY{n}{numProc} \PY{o}{=} \PY{n}{numProc}
                \PY{n+nb+bp}{self}\PY{o}{.}\PY{n}{tiempoTotal} \PY{o}{=} \PY{l+m+mi}{0}
                \PY{n+nb+bp}{self}\PY{o}{.}\PY{n}{tupla} \PY{o}{=} \PY{p}{[}\PY{p}{]}
                \PY{n}{resultados}\PY{o}{.}\PY{n}{append}\PY{p}{(}\PY{n+nb+bp}{self}\PY{o}{.}\PY{n}{tupla}\PY{p}{)}
        
            \PY{k}{def} \PY{n+nf}{crearProceso}\PY{p}{(}\PY{n+nb+bp}{self}\PY{p}{,} \PY{n}{i}\PY{p}{)}\PY{p}{:}
                \PY{n}{p} \PY{o}{=} \PY{n}{Proceso}\PY{p}{(}\PY{n}{i}\PY{p}{,} \PY{n}{np}\PY{o}{.}\PY{n}{random}\PY{o}{.}\PY{n}{randint}\PY{p}{(}\PY{l+m+mi}{1}\PY{p}{,}\PY{l+m+mi}{100}\PY{p}{)}\PY{p}{,} \PY{n+nb+bp}{True}\PY{p}{)}
                \PY{k}{print} \PY{l+s+s2}{\PYZdq{}}\PY{l+s+s2}{Proceso }\PY{l+s+si}{\PYZpc{}d}\PY{l+s+s2}{ \PYZhy{}\PYZhy{}\PYZgt{} Tiempo = }\PY{l+s+si}{\PYZpc{}d}\PY{l+s+s2}{\PYZdq{}}\PY{o}{\PYZpc{}}\PY{p}{(}\PY{n}{p}\PY{o}{.}\PY{n}{getIdProc}\PY{p}{(}\PY{p}{)}\PY{p}{,} \PY{n}{p}\PY{o}{.}\PY{n}{getTiempo}\PY{p}{(}\PY{p}{)}\PY{p}{)}
                \PY{n+nb+bp}{self}\PY{o}{.}\PY{n}{tupla}\PY{o}{.}\PY{n}{append}\PY{p}{(}\PY{n}{p}\PY{o}{.}\PY{n}{getTiempo}\PY{p}{(}\PY{p}{)}\PY{p}{)}
                \PY{k}{return} \PY{n}{p}
        
            \PY{k}{def} \PY{n+nf}{crearCola}\PY{p}{(}\PY{n+nb+bp}{self}\PY{p}{)}\PY{p}{:}
                \PY{n+nb+bp}{self}\PY{o}{.}\PY{n}{cola} \PY{o}{=} \PY{n}{Cola}\PY{p}{(}\PY{p}{)}
                \PY{k}{for} \PY{n}{i} \PY{o+ow}{in} \PY{n+nb}{range}\PY{p}{(}\PY{n+nb+bp}{self}\PY{o}{.}\PY{n}{numProc}\PY{p}{)}\PY{p}{:}
                    \PY{n+nb+bp}{self}\PY{o}{.}\PY{n}{cola}\PY{o}{.}\PY{n}{agregar}\PY{p}{(}\PY{n+nb+bp}{self}\PY{o}{.}\PY{n}{crearProceso}\PY{p}{(}\PY{n}{i}\PY{o}{+}\PY{l+m+mi}{1}\PY{p}{)}\PY{p}{)}
        
            \PY{k}{def} \PY{n+nf}{procesar}\PY{p}{(}\PY{n+nb+bp}{self}\PY{p}{)}\PY{p}{:}
                \PY{n+nb+bp}{self}\PY{o}{.}\PY{n}{crearCola}\PY{p}{(}\PY{p}{)}
                \PY{k}{print} \PY{l+s+s2}{\PYZdq{}}\PY{l+s+se}{\PYZbs{}n}\PY{l+s+s2}{Proceso }\PY{l+s+se}{\PYZbs{}t}\PY{l+s+s2}{ T. restante }\PY{l+s+se}{\PYZbs{}t}\PY{l+s+s2}{ Total}\PY{l+s+s2}{\PYZdq{}}
                \PY{n}{turnos} \PY{o}{=} \PY{l+m+mi}{0}
                \PY{k}{while} \PY{n+nb+bp}{self}\PY{o}{.}\PY{n}{cola}\PY{o}{.}\PY{n}{estaVacia}\PY{p}{(}\PY{p}{)} \PY{o}{==} \PY{n+nb+bp}{False}\PY{p}{:}
                    \PY{n}{turnos} \PY{o}{+}\PY{o}{=} \PY{l+m+mi}{1}
                    \PY{n}{q} \PY{o}{=} \PY{n+nb+bp}{self}\PY{o}{.}\PY{n}{cola}\PY{o}{.}\PY{n}{getFirst}\PY{p}{(}\PY{p}{)}\PY{o}{.}\PY{n}{ejecutar}\PY{p}{(}\PY{n+nb+bp}{self}\PY{o}{.}\PY{n}{quantum}\PY{p}{)}
                    \PY{n+nb+bp}{self}\PY{o}{.}\PY{n}{tiempoTotal} \PY{o}{+}\PY{o}{=} \PY{n}{q}
                    \PY{k}{print} \PY{l+s+s2}{\PYZdq{}}\PY{l+s+si}{\PYZpc{}d}\PY{l+s+s2}{ }\PY{l+s+se}{\PYZbs{}t}\PY{l+s+se}{\PYZbs{}t}\PY{l+s+s2}{ }\PY{l+s+si}{\PYZpc{}d}\PY{l+s+s2}{ }\PY{l+s+se}{\PYZbs{}t}\PY{l+s+se}{\PYZbs{}t}\PY{l+s+s2}{ }\PY{l+s+si}{\PYZpc{}d}\PY{l+s+s2}{\PYZdq{}}\PY{o}{\PYZpc{}}\PY{p}{(}\PY{n+nb+bp}{self}\PY{o}{.}\PY{n}{cola}\PY{o}{.}\PY{n}{getFirst}\PY{p}{(}\PY{p}{)}\PY{o}{.}\PY{n}{getIdProc}\PY{p}{(}\PY{p}{)}\PY{p}{,} \PY{n+nb+bp}{self}\PY{o}{.}\PY{n}{cola}\PY{o}{.}\PY{n}{getFirst}\PY{p}{(}\PY{p}{)}\PY{o}{.}\PY{n}{getTiempo}\PY{p}{(}\PY{p}{)}\PY{p}{,} \PY{n+nb+bp}{self}\PY{o}{.}\PY{n}{tiempoTotal}\PY{p}{)}
        
                    \PY{k}{if} \PY{n+nb+bp}{self}\PY{o}{.}\PY{n}{cola}\PY{o}{.}\PY{n}{getFirst}\PY{p}{(}\PY{p}{)}\PY{o}{.}\PY{n}{tiempo} \PY{o}{\PYZgt{}} \PY{l+m+mi}{0}\PY{p}{:}
                        \PY{n+nb+bp}{self}\PY{o}{.}\PY{n}{cola}\PY{o}{.}\PY{n}{agregar}\PY{p}{(}\PY{n+nb+bp}{self}\PY{o}{.}\PY{n}{cola}\PY{o}{.}\PY{n}{getFirst}\PY{p}{(}\PY{p}{)}\PY{p}{)}
        
                    \PY{n+nb+bp}{self}\PY{o}{.}\PY{n}{cola}\PY{o}{.}\PY{n}{avanzar}\PY{p}{(}\PY{p}{)}\PY{p}{;}
        
                \PY{n+nb+bp}{self}\PY{o}{.}\PY{n}{tupla}\PY{o}{.}\PY{n}{append}\PY{p}{(}\PY{n}{turnos}\PY{p}{)}
                \PY{k}{return} \PY{n}{turnos}
        
        \PY{k}{class} \PY{n+nc}{Cola}\PY{p}{:}
            \PY{n}{count} \PY{o}{=} \PY{o}{\PYZhy{}}\PY{l+m+mi}{1}
            \PY{k}{def} \PY{n+nf+fm}{\PYZus{}\PYZus{}init\PYZus{}\PYZus{}}\PY{p}{(}\PY{n+nb+bp}{self}\PY{p}{)}\PY{p}{:}
                \PY{n+nb+bp}{self}\PY{o}{.}\PY{n}{items} \PY{o}{=} \PY{p}{[}\PY{p}{]}
        
            \PY{k}{def} \PY{n+nf}{getFirst}\PY{p}{(}\PY{n+nb+bp}{self}\PY{p}{)}\PY{p}{:}
                \PY{k}{return} \PY{n+nb+bp}{self}\PY{o}{.}\PY{n}{items}\PY{p}{[}\PY{l+m+mi}{0}\PY{p}{]}
        
        
            \PY{k}{def} \PY{n+nf}{getItems}\PY{p}{(}\PY{n+nb+bp}{self}\PY{p}{)}\PY{p}{:}
                \PY{k}{return} \PY{n+nb+bp}{self}\PY{o}{.}\PY{n}{items}
        
            \PY{k}{def} \PY{n+nf}{getItem}\PY{p}{(}\PY{n+nb+bp}{self}\PY{p}{,} \PY{n}{ix}\PY{p}{)}\PY{p}{:}
                \PY{k}{return} \PY{n+nb+bp}{self}\PY{o}{.}\PY{n}{items}\PY{p}{[}\PY{n}{ix}\PY{p}{]}
        
            \PY{k}{def} \PY{n+nf}{estaVacia}\PY{p}{(}\PY{n+nb+bp}{self}\PY{p}{)}\PY{p}{:}
                \PY{k}{if} \PY{n}{Cola}\PY{o}{.}\PY{n}{count} \PY{o}{\PYZlt{}} \PY{l+m+mi}{0}\PY{p}{:}
                    \PY{k}{return} \PY{n+nb+bp}{True}
                \PY{k}{else}\PY{p}{:}
                    \PY{k}{return} \PY{n+nb+bp}{False}
        
            \PY{k}{def} \PY{n+nf}{agregar}\PY{p}{(}\PY{n+nb+bp}{self}\PY{p}{,} \PY{n}{item}\PY{p}{)}\PY{p}{:}
                \PY{n}{Cola}\PY{o}{.}\PY{n}{count} \PY{o}{+}\PY{o}{=} \PY{l+m+mi}{1}
                \PY{n+nb+bp}{self}\PY{o}{.}\PY{n}{items}\PY{o}{.}\PY{n}{append}\PY{p}{(}\PY{n}{item}\PY{p}{)}
        
            \PY{k}{def} \PY{n+nf}{avanzar}\PY{p}{(}\PY{n+nb+bp}{self}\PY{p}{)}\PY{p}{:}
                \PY{n}{Cola}\PY{o}{.}\PY{n}{count} \PY{o}{\PYZhy{}}\PY{o}{=} \PY{l+m+mi}{1}
                \PY{k}{return} \PY{n+nb+bp}{self}\PY{o}{.}\PY{n}{items}\PY{o}{.}\PY{n}{pop}\PY{p}{(}\PY{l+m+mi}{0}\PY{p}{)}
        
            \PY{k}{def} \PY{n+nf}{tamano}\PY{p}{(}\PY{n+nb+bp}{self}\PY{p}{)}\PY{p}{:}
                \PY{k}{return} \PY{n+nb}{len}\PY{p}{(}\PY{n+nb+bp}{self}\PY{o}{.}\PY{n}{items}\PY{p}{)}
\end{Verbatim}


    \begin{Verbatim}[commandchars=\\\{\}]
{\color{incolor}In [{\color{incolor} }]:} \PY{n}{n} \PY{o}{=} \PY{n+nb}{int}\PY{p}{(}\PY{n+nb}{raw\PYZus{}input}\PY{p}{(}\PY{l+s+s1}{\PYZsq{}}\PY{l+s+s1}{Cuantos procesos?}\PY{l+s+se}{\PYZbs{}t}\PY{l+s+s1}{\PYZsq{}}\PY{p}{)}\PY{p}{)}
        \PY{n}{q} \PY{o}{=} \PY{n+nb}{int}\PY{p}{(}\PY{n+nb}{raw\PYZus{}input}\PY{p}{(}\PY{l+s+s1}{\PYZsq{}}\PY{l+s+s1}{Cual es la duracion del quantum?}\PY{l+s+se}{\PYZbs{}t}\PY{l+s+s1}{\PYZsq{}}\PY{p}{)}\PY{p}{)}
        
        \PY{n}{turnos} \PY{o}{=} \PY{n}{RoundRobin}\PY{p}{(}\PY{n}{q}\PY{p}{,} \PY{n}{n}\PY{p}{)}\PY{o}{.}\PY{n}{procesar}\PY{p}{(}\PY{p}{)}
        \PY{k}{print} \PY{l+s+s2}{\PYZdq{}}\PY{l+s+s2}{Se completo en un total de }\PY{l+s+si}{\PYZpc{}d}\PY{l+s+s2}{ turnos}\PY{l+s+se}{\PYZbs{}n}\PY{l+s+s2}{\PYZdq{}}\PY{o}{\PYZpc{}}\PY{p}{(}\PY{n}{turnos}\PY{p}{)}
\end{Verbatim}


    \subsection{Resultado}\label{resultado}

Con el propósito de evaluar el funcionamiento de este programa se
tomaron únicamente 2 procesos, con un quantum fijo de 20, y se
realizaron 50 pruebas, donde el tiempo de cada uno de los procesos (que
es asignado aleatoreamente) y el número de turnos totales se registraron
en la lista resultados.

    \begin{Verbatim}[commandchars=\\\{\}]
{\color{incolor}In [{\color{incolor}12}]:} \PY{k+kn}{from} \PY{n+nn}{IPython.display} \PY{k+kn}{import} \PY{n}{clear\PYZus{}output}
         \PY{n}{n} \PY{o}{=} \PY{l+m+mi}{2} \PY{c+c1}{\PYZsh{}int(raw\PYZus{}input(\PYZsq{}Cuantos procesos?\PYZbs{}t\PYZsq{}))}
         \PY{n}{q} \PY{o}{=} \PY{l+m+mi}{20} \PY{c+c1}{\PYZsh{}int(raw\PYZus{}input(\PYZsq{}Cual es la duracion del quantum?\PYZbs{}t\PYZsq{}))}
         \PY{k}{for} \PY{n}{i} \PY{o+ow}{in} \PY{n+nb}{range}\PY{p}{(}\PY{l+m+mi}{50}\PY{p}{)}\PY{p}{:}
             \PY{n}{clear\PYZus{}output}\PY{p}{(}\PY{p}{)} \PY{c+c1}{\PYZsh{}Asegura que únicamente el último se muestre }
             \PY{n}{turnos} \PY{o}{=} \PY{n}{RoundRobin}\PY{p}{(}\PY{n}{q}\PY{p}{,} \PY{n}{n}\PY{p}{)}\PY{o}{.}\PY{n}{procesar}\PY{p}{(}\PY{p}{)}
             \PY{k}{print} \PY{l+s+s2}{\PYZdq{}}\PY{l+s+s2}{Se completo en un total de }\PY{l+s+si}{\PYZpc{}d}\PY{l+s+s2}{ turnos}\PY{l+s+se}{\PYZbs{}n}\PY{l+s+s2}{\PYZdq{}}\PY{o}{\PYZpc{}}\PY{p}{(}\PY{n}{turnos}\PY{p}{)}
\end{Verbatim}


    \begin{Verbatim}[commandchars=\\\{\}]
Proceso 1 --> Tiempo = 5
Proceso 2 --> Tiempo = 79

Proceso 	 T. restante 	 Total
1 		 0 		 5
2 		 59 		 25
2 		 39 		 45
2 		 19 		 65
2 		 0 		 84
Se completo en un total de 5 turnos


    \end{Verbatim}

    \begin{Verbatim}[commandchars=\\\{\}]
{\color{incolor}In [{\color{incolor}22}]:} \PY{k+kn}{import} \PY{n+nn}{pandas} \PY{k+kn}{as} \PY{n+nn}{pd}
         \PY{c+c1}{\PYZsh{}print np.array(resultados)[:10]}
         \PY{n}{df} \PY{o}{=} \PY{n}{pd}\PY{o}{.}\PY{n}{DataFrame}\PY{p}{(}\PY{n}{np}\PY{o}{.}\PY{n}{array}\PY{p}{(}\PY{n}{resultados}\PY{p}{)}\PY{p}{,} \PY{n}{columns}\PY{o}{=}\PY{p}{[}\PY{l+s+s1}{\PYZsq{}}\PY{l+s+s1}{Proc\PYZus{}1}\PY{l+s+s1}{\PYZsq{}}\PY{p}{,} \PY{l+s+s1}{\PYZsq{}}\PY{l+s+s1}{Proc\PYZus{}2}\PY{l+s+s1}{\PYZsq{}}\PY{p}{,} \PY{l+s+s1}{\PYZsq{}}\PY{l+s+s1}{Turnos}\PY{l+s+s1}{\PYZsq{}}\PY{p}{]}\PY{p}{)}
         \PY{n}{df}\PY{o}{.}\PY{n}{head}\PY{p}{(}\PY{p}{)}
\end{Verbatim}


\begin{Verbatim}[commandchars=\\\{\}]
{\color{outcolor}Out[{\color{outcolor}22}]:}    Proc\_1  Proc\_2  Turnos
         0      30      92       7
         1      97      83      10
         2      38      11       3
         3      61      74       8
         4      78      36       6
\end{Verbatim}
            
    \begin{Verbatim}[commandchars=\\\{\}]
{\color{incolor}In [{\color{incolor}14}]:} \PY{k}{print} \PY{p}{(}\PY{l+s+s1}{\PYZsq{}}\PY{l+s+s1}{Proceso 1 min y max: }\PY{l+s+si}{\PYZpc{}d}\PY{l+s+s1}{, }\PY{l+s+si}{\PYZpc{}d}\PY{l+s+s1}{\PYZsq{}}\PY{p}{)}\PY{o}{\PYZpc{}}\PY{p}{(}\PY{n}{np}\PY{o}{.}\PY{n}{min}\PY{p}{(}\PY{p}{[}\PY{n}{resultados}\PY{p}{[}\PY{n}{i}\PY{p}{]}\PY{p}{[}\PY{l+m+mi}{0}\PY{p}{]} \PY{k}{for} \PY{n}{i} \PY{o+ow}{in} \PY{n+nb}{range}\PY{p}{(}\PY{n+nb}{len}\PY{p}{(}\PY{n}{resultados}\PY{p}{)}\PY{p}{)}\PY{p}{]}\PY{p}{)}\PY{p}{,}\PY{n}{np}\PY{o}{.}\PY{n}{max}\PY{p}{(}\PY{p}{[}\PY{n}{resultados}\PY{p}{[}\PY{n}{i}\PY{p}{]}\PY{p}{[}\PY{l+m+mi}{0}\PY{p}{]} \PY{k}{for} \PY{n}{i} \PY{o+ow}{in} \PY{n+nb}{range}\PY{p}{(}\PY{n+nb}{len}\PY{p}{(}\PY{n}{resultados}\PY{p}{)}\PY{p}{)}\PY{p}{]}\PY{p}{)}\PY{p}{)}
         \PY{k}{print} \PY{p}{(}\PY{l+s+s1}{\PYZsq{}}\PY{l+s+s1}{Proceso 2 min y max: }\PY{l+s+si}{\PYZpc{}d}\PY{l+s+s1}{, }\PY{l+s+si}{\PYZpc{}d}\PY{l+s+s1}{\PYZsq{}}\PY{p}{)}\PY{o}{\PYZpc{}}\PY{p}{(}\PY{n}{np}\PY{o}{.}\PY{n}{min}\PY{p}{(}\PY{p}{[}\PY{n}{resultados}\PY{p}{[}\PY{n}{i}\PY{p}{]}\PY{p}{[}\PY{l+m+mi}{1}\PY{p}{]} \PY{k}{for} \PY{n}{i} \PY{o+ow}{in} \PY{n+nb}{range}\PY{p}{(}\PY{n+nb}{len}\PY{p}{(}\PY{n}{resultados}\PY{p}{)}\PY{p}{)}\PY{p}{]}\PY{p}{)}\PY{p}{,}\PY{n}{np}\PY{o}{.}\PY{n}{max}\PY{p}{(}\PY{p}{[}\PY{n}{resultados}\PY{p}{[}\PY{n}{i}\PY{p}{]}\PY{p}{[}\PY{l+m+mi}{1}\PY{p}{]} \PY{k}{for} \PY{n}{i} \PY{o+ow}{in} \PY{n+nb}{range}\PY{p}{(}\PY{n+nb}{len}\PY{p}{(}\PY{n}{resultados}\PY{p}{)}\PY{p}{)}\PY{p}{]}\PY{p}{)}\PY{p}{)}
         \PY{k}{print} \PY{p}{(}\PY{l+s+s1}{\PYZsq{}}\PY{l+s+s1}{Tiempo total promedio: }\PY{l+s+si}{\PYZpc{}.3f}\PY{l+s+s1}{\PYZsq{}}\PY{p}{)}\PY{o}{\PYZpc{}}\PY{p}{(}\PY{n}{np}\PY{o}{.}\PY{n}{mean}\PY{p}{(}\PY{p}{[}\PY{p}{[}\PY{n}{resultados}\PY{p}{[}\PY{n}{i}\PY{p}{]}\PY{p}{[}\PY{l+m+mi}{1}\PY{p}{]} \PY{o}{+} \PY{n}{resultados}\PY{p}{[}\PY{n}{i}\PY{p}{]}\PY{p}{[}\PY{l+m+mi}{1}\PY{p}{]}\PY{p}{]} \PY{k}{for} \PY{n}{i} \PY{o+ow}{in} \PY{n+nb}{range}\PY{p}{(}\PY{n+nb}{len}\PY{p}{(}\PY{n}{resultados}\PY{p}{)}\PY{p}{)}\PY{p}{]}\PY{p}{)}\PY{p}{)}
         \PY{k}{print} \PY{p}{(}\PY{l+s+s1}{\PYZsq{}}\PY{l+s+s1}{Proceso 1 promedio: }\PY{l+s+si}{\PYZpc{}.3f}\PY{l+s+s1}{\PYZsq{}}\PY{p}{)}\PY{o}{\PYZpc{}}\PY{p}{(}\PY{n}{np}\PY{o}{.}\PY{n}{mean}\PY{p}{(}\PY{p}{[}\PY{n}{resultados}\PY{p}{[}\PY{n}{i}\PY{p}{]}\PY{p}{[}\PY{l+m+mi}{0}\PY{p}{]} \PY{k}{for} \PY{n}{i} \PY{o+ow}{in} \PY{n+nb}{range}\PY{p}{(}\PY{n+nb}{len}\PY{p}{(}\PY{n}{resultados}\PY{p}{)}\PY{p}{)}\PY{p}{]}\PY{p}{)}\PY{p}{)}
         \PY{k}{print} \PY{p}{(}\PY{l+s+s1}{\PYZsq{}}\PY{l+s+s1}{Proceso 2 promedio: }\PY{l+s+si}{\PYZpc{}.3f}\PY{l+s+s1}{\PYZsq{}}\PY{p}{)}\PY{o}{\PYZpc{}}\PY{p}{(}\PY{n}{np}\PY{o}{.}\PY{n}{mean}\PY{p}{(}\PY{p}{[}\PY{n}{resultados}\PY{p}{[}\PY{n}{i}\PY{p}{]}\PY{p}{[}\PY{l+m+mi}{1}\PY{p}{]} \PY{k}{for} \PY{n}{i} \PY{o+ow}{in} \PY{n+nb}{range}\PY{p}{(}\PY{n+nb}{len}\PY{p}{(}\PY{n}{resultados}\PY{p}{)}\PY{p}{)}\PY{p}{]}\PY{p}{)}\PY{p}{)}
         \PY{k}{print} \PY{p}{(}\PY{l+s+s1}{\PYZsq{}}\PY{l+s+s1}{Turnos promedio: }\PY{l+s+si}{\PYZpc{}.3f}\PY{l+s+s1}{\PYZsq{}}\PY{p}{)}\PY{o}{\PYZpc{}}\PY{p}{(}\PY{n}{np}\PY{o}{.}\PY{n}{mean}\PY{p}{(}\PY{p}{[}\PY{n}{resultados}\PY{p}{[}\PY{n}{i}\PY{p}{]}\PY{p}{[}\PY{l+m+mi}{2}\PY{p}{]} \PY{k}{for} \PY{n}{i} \PY{o+ow}{in} \PY{n+nb}{range}\PY{p}{(}\PY{n+nb}{len}\PY{p}{(}\PY{n}{resultados}\PY{p}{)}\PY{p}{)}\PY{p}{]}\PY{p}{)}\PY{p}{)}
\end{Verbatim}


    \begin{Verbatim}[commandchars=\\\{\}]
Proceso 1 min y max: 1, 98
Proceso 2 min y max: 1, 99
Tiempo total promedio: 102.580
Proceso 1 promedio: 51.120
Proceso 2 promedio: 51.290
Turnos promedio: 5.970

    \end{Verbatim}

    Se pudo observar que en ninguno de los casos se generó innanición en
ninguno de los procesos, y el tiempo de ejecución total corresponde a la
suma de los tiempos individuaes de cada proceso. Esto se puede observar
más claramente en la siguiente gráfica siguiente, realizada para el
último test, presente en resultados.

    \subsection{Conclusiones}\label{conclusiones}

    \subsection{Fuentes de Información}\label{fuentes-de-informaciuxf3n}


    % Add a bibliography block to the postdoc
    
    
    
    \end{document}
